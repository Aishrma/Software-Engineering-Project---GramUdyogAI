\documentclass{article}
\usepackage{geometry}
\geometry{a4paper, margin=1in}
\usepackage{fancyhdr}
\usepackage{enumitem}
\usepackage{hyperref}
\usepackage{graphicx} % Required for including images

\hypersetup{
    colorlinks=true,
    linkcolor=blue,
    filecolor=magenta,      
    urlcolor=cyan,
}

\pagestyle{fancy}
\fancyhf{}
\rhead{GramUdyogAI}
\lhead{SRS Document}
\cfoot{\thepage}

\title{Software Requirements Specification for GramUdyogAI}
\author{Generated by GitHub Copilot}
\date{\today}
 
\begin{document}

\maketitle

\tableofcontents

\newpage

\section{INTRODUCTION}

\subsection{Purpose}
The purpose of this document is to provide a detailed description of the GramUdyogAI system. It specifies the functional and non-functional requirements of the system, serving as a comprehensive guide for the development, testing, and future evolution of the project. This SRS aims to address the unique challenges faced by rural entrepreneurs, such as limited access to information, language barriers, and lack of awareness about government schemes, by outlining a clear set of features and capabilities.

\subsection{Intended Audience and Reading Suggestions}
This document is intended for a wide range of stakeholders, including:
\begin{itemize}
    \item \textbf{Project Managers:} To oversee the project lifecycle and ensure requirements are met.
    \item \textbf{Developers:} To understand the technical specifications and build the system accordingly.
    \item \textbf{Testers:} To create test cases and verify that the system functions as expected.
    \item \textbf{Investors and Partners:} To understand the product's scope, vision, and potential impact.
\end{itemize}
Readers should have a basic understanding of software development concepts. It is recommended to read the document sequentially, although specific sections can be consulted for targeted information.

\subsection{Product Scope}
GramUdyogAI is an AI-powered platform designed to empower rural entrepreneurs by providing a suite of tools and resources. The core functionalities include:
\begin{itemize}
    \item Personalized business suggestions tailored to rural markets.
    \item A centralized and easily accessible database of government schemes.
    \item Conversion of educational YouTube content into audio summaries in local languages.
    \item Networking opportunities through a dedicated job and project board.
\end{itemize}
The platform aims to bridge the opportunity gap between rural and urban areas by leveraging technology to deliver critical information and services.

\section{OVERALL DESCRIPTION}

\subsection{Product Perspective}
GramUdyogAI is a new, self-contained product designed to be a comprehensive digital companion for rural entrepreneurs. It stands apart from generic business tools by focusing specifically on the needs and context of the rural ecosystem. The platform integrates AI-driven insights with practical resources, creating a unique value proposition for its target audience. It is envisioned as a catalyst for local economic growth and self-reliance.

\subsection{Product Features}
The key features of GramUdyogAI include:
\begin{itemize}
    \item \textbf{AI-Powered Business Suggestions:} Users can input their skills, available resources, and location to receive tailored business recommendations. The AI considers local market demand and resource availability to suggest viable and sustainable ventures.
    \item \textbf{Government Schemes Database:} A comprehensive and up-to-date repository of government programs, subsidies, and support initiatives. The database is searchable and filterable, making it easy for users to find relevant schemes.
    \item \textbf{YouTube Audio Summary:} This feature allows users to convert any YouTube video into a concise audio summary in their preferred local language, overcoming literacy and language barriers to accessing educational content.
    \item \textbf{Networking and Job Board:} A platform for users to connect with other professionals, post job openings, and find relevant opportunities within their region, fostering a collaborative ecosystem.
    \item \textbf{Multilingual Support:} The entire platform is designed to be accessible in multiple regional languages, ensuring inclusivity and broad reach.
    \item \textbf{Secure Multi-User Authentication:} A robust authentication system supports various user types (Individual, Company, NGO, Investor), each with tailored access and features, secured with modern standards like JWT and password hashing.
\end{itemize}

\subsection{Operating Environment}
The system is a web-based application designed to operate in a variety of environments:
\begin{itemize}
    \item \textbf{Client-Side:} Accessible on any modern web browser (e.g., Chrome, Firefox, Safari) on desktops, smartphones, and tablets. The frontend is a Progressive Web App (PWA) to support offline access and low-data usage.
    \item \textbf{Server-Side:} The backend is hosted on Railway, utilizing a containerized microservices architecture for scalability. It runs on a Python environment with the FastAPI framework.
    \item \textbf{Deployment:} The frontend is deployed on Vercel for global availability and performance, while the backend infrastructure is managed by Railway.
\end{itemize}

\section{SYSTEM ARCHITECTURE AND DESIGN}

\subsection{Class Diagram}
The class diagram in Figure \ref{fig:class_diagram} illustrates the static structure of the GramUdyogAI system, showing the main classes and their relationships.

\begin{figure}[h!]
    \centering
    \includegraphics[width=0.9\textwidth]{images/Class_Diagram.png}
    \caption{System Class Diagram}
    \label{fig:class_diagram}
\end{figure}

\subsection{Use Case Diagram}
The use case diagram in Figure \ref{fig:usecase_diagram} shows the interactions between the users (actors) and the main features of the system.

\begin{figure}[h!]
    \centering
    \includegraphics[width=0.9\textwidth]{images/UseCase_Diagram.png}
    \caption{System Use Case Diagram}
    \label{fig:usecase_diagram}
\end{figure}

\subsection{Sequence Diagram}
The sequence diagram in Figure \ref{fig:sequence_diagram} details the step-by-step interactions for the "Get AI Business Suggestion" feature, from the user's action to the final response.

\begin{figure}[h!]
    \centering
    \includegraphics[width=0.9\textwidth]{images/Sequence_Diagram.png}
    \caption{Sequence Diagram for AI Business Suggestion}
    \label{fig:sequence_diagram}
\end{figure}


\section{EXTERNAL INTERFACE REQUIREMENTS}

\subsection{User Interfaces}
The user interface is a responsive web application designed with a mobile-first approach. Key UI characteristics include:
\begin{itemize}
    \item \textbf{Simplicity and Clarity:} A clean and intuitive layout to ensure ease of use for non-technical users.
    \item \textbf{Accessibility:} Adherence to accessibility standards, including support for text-to-speech and high-contrast themes.
    \item \textbf{Responsiveness:} The UI adapts seamlessly to different screen sizes, from basic smartphones to large desktop monitors.
    \item \textbf{Low-Bandwidth Optimization:} The application is optimized to perform well even on slow or intermittent internet connections.
\end{itemize}

\subsection{Hardware Interfaces}
No specialized hardware is required. The system is accessible through standard computing devices with the following:
\begin{itemize}
    \item A screen for display.
    \item A keyboard, mouse, or touch screen for input.
    \item A microphone and speakers for audio features (optional but recommended).
    \item An active internet connection.
\end{itemize}

\subsection{Software Interfaces}
The system interacts with several software components and external services:
\begin{itemize}
    \item \textbf{Frontend Framework:} React with TypeScript for building a robust and maintainable user interface.
    \item \textbf{Backend Framework:} Python with FastAPI for creating efficient and scalable APIs.
    \item \textbf{AI/ML Services:} Integration with external APIs from OpenAI and Groq for natural language processing, business suggestions, and content summarization.
    \item \textbf{Database:} A relational or NoSQL database (e.g., PostgreSQL, MongoDB) will be used to store user profiles, application data, and generated content.
    \item \textbf{Internationalization:} The i18next library is used for managing multilingual support on the frontend.
\end{itemize}

\section{SYSTEM FEATURES}

\subsection{AI Business Suggestions}
\begin{itemize}
    \item \textbf{Description:} This feature provides users with personalized business ideas. The user inputs their skills (e.g., farming, tailoring), available resources (e.g., land, equipment), and location. The AI model processes this information to generate a list of suitable business recommendations.
    \item \textbf{User Interaction:} The user fills out a simple form. The system displays a list of suggestions with brief descriptions and potential benefits.
    \item \textbf{Expected Outcome:} The user receives actionable and context-aware business ideas that they can explore further.
\end{itemize}

\subsection{Government Schemes Database}
\begin{itemize}
    \item \textbf{Description:} This feature offers a centralized portal to access information on various government schemes. It includes details on eligibility, application processes, and benefits.
    \item \textbf{User Interaction:} Users can search for schemes using keywords or filter them by category (e.g., agriculture, education).
    \item \textbf{Expected Outcome:} Users can easily find and understand government support available to them, simplifying the application process.
\end{itemize}

\subsection{YouTube Audio Summary}
\begin{itemize}
    \item \textbf{Description:} This feature makes video content more accessible by converting it into audio summaries. Users can paste a YouTube URL, and the system generates a summary and converts it to speech in a selected local language.
    \item \textbf{User Interaction:} The user provides a YouTube link and selects a language. The system presents an audio player to listen to the summary.
    \item \textbf{Expected Outcome:} Users can consume educational and informational content without needing to watch a video or be literate in the video's original language.
\end{itemize}

\subsection{Networking and Job Board}
\begin{itemize}
    \item \textbf{Description:} A dedicated section for professional networking, allowing users to post and discover job opportunities, collaborations, and projects.
    \item \textbf{User Interaction:} Users can create a professional profile, post job listings, and apply for opportunities.
    \item \textbf{Expected Outcome:} A connected community of rural entrepreneurs, professionals, and organizations, leading to increased collaboration and employment.
\end{itemize}

\subsection{Multilingual Support}
\begin{itemize}
    \item \textbf{Description:} The platform's text and features are available in multiple regional languages.
    \item \textbf{User Interaction:} Users can select their preferred language from a dropdown menu, and the entire interface will switch to that language.
    \item \textbf{Expected Outcome:} The platform is accessible and usable for a diverse, non-English-speaking audience.
\end{itemize}

\subsection{Secure Authentication}
\begin{itemize}
    \item \textbf{Description:} A secure system for user registration, login, and session management. It supports different user roles (Individual, Company, etc.) with distinct permissions.
    \item \textbf{User Interaction:} Users can register for an account, log in with their credentials, and reset their password.
    \item \textbf{Expected Outcome:} User data is protected, and access to features is controlled based on the user's role and permissions.
\end{itemize}

\section{OTHER NON-FUNCTIONAL REQUIREMENTS}

\subsection{Performance Requirements}
\begin{itemize}
    \item \textbf{Response Time:} Core API responses should be completed within 2 seconds under normal load. AI-intensive tasks may have longer processing times, which should be communicated to the user with loading indicators.
    \item \textbf{Page Load:} Key pages should load within 3-5 seconds on a 3G network connection.
    \item \textbf{Concurrency:} The system should support at least 100 concurrent users without significant degradation in performance.
\end{itemize}

\subsection{Safety Requirements}
The system is an informational and networking platform and does not control any safety-critical equipment or processes. Therefore, no specific safety requirements beyond standard data protection are identified. The system must not provide financial or legal advice, and disclaimers should be in place to that effect.

\subsection{Security Requirements}
\begin{itemize}
    \item \textbf{Data Encryption:} All communication between the client and server must be encrypted using TLS/SSL. Sensitive user data stored in the database should be encrypted at rest.
    \item \textbf{Authentication and Authorization:} The system must implement robust authentication with password hashing (e.g., bcrypt) and JWT for session management. Access control must be enforced to prevent unauthorized access to data and features.
    \item \textbf{Vulnerability Protection:} The application should be protected against common web vulnerabilities, including SQL injection, Cross-Site Scripting (XSS), and Cross-Site Request Forgery (CSRF).
\end{itemize}

\subsection{Software Quality Attributes}
\begin{itemize}
    \item \textbf{Usability:} The system must be highly intuitive and require minimal training for the target audience. The user journey should be simple and goal-oriented.
    \item \textbf{Scalability:} The architecture must be designed to scale horizontally. The use of microservices and serverless deployment is intended to accommodate a growing user base and increasing data load.
    \item \textbf{Reliability:} The system should aim for an uptime of 99.5\%. It should be resilient to failures, with graceful error handling and recovery mechanisms in place.
    \item \textbf{Maintainability:} The code should be well-documented, modular, and follow consistent coding standards to facilitate future updates and maintenance.
\end{itemize}

\end{document}

